\documentclass{article}
\usepackage[czech]{babel}
\usepackage[utf8]{inputenc}
\usepackage[T1]{fontenc}
\usepackage[left=2cm, text={17cm, 24cm}, top=2cm]{geometry}
\usepackage{graphicx}
\usepackage{amsmath}
\usepackage{xcolor}
\usepackage[european]{circuitikz}
\usepackage{amssymb}
\title{IEL -- Semestrálny projekt}
\author{Tomáš Ďuriš}
\date{\today}
\begin{document}
	\begin{titlepage}
		\begin{center}	
			\Huge{Vysoké učení technické v Brně}\\
			\huge{Fakulta informačných technologií}\\
			\vspace{1cm}
			\huge{Elektronika pre informačné technológie}
			\vspace{3cm}
			\huge{2018/2019}\\
			\vspace{2cm}
			\huge{\textbf{Semestrálny projekt}}\\
			\vspace{12cm}
	    	\Large{Tomáš Ďuriš (xduris05)~~~~~~~~~~~~~~~~~~ Brno 19.12 2018}
		\end{center}	
	\end{titlepage}
\section*{1.}
\large{Zadanie:}
Stanovte $U_{R3}$ a $I_{R3}$. Použite metódu postupného zjednodušovania obvodu
\begin{table}[ht]
    \centering
    \begin{tabular}{|[|c|c|c|c|c|c|c|c|c|c|c|c|c|c]}
    \hline
    sk.&U_1 [V]&U_2 [V]&R_1 [\Omega]&R_2 [\Omega]&R_3 [\Omega]&R_4 [\Omega]&R_5 [\Omega]&R_6 [\Omega]&R_7 [\Omega]&R_8 [\Omega]\\
    \hline
    D&105&85&420&980&330&280&310&710&240&200\\
    \hline
    \end{tabular}
\end{table}
\newline
\begin{center}
    \begin{circuitikz}
        \draw
        (0,4) to[dcvsource, v_=$U_1$] (0,2)
        (0,2) to[dcvsource, v_=$U_2$] (0,0)
        (0,4) to[short, -*] (1,4)
        (1,4) to[R, *-, l=$R_1$] (4,4)
        (4,4) to[R, -*, l=$R_2$] (7,4)
        (1,4) to[short, *-] (1,2)
        (1,2) to[R, -*, l=$R_3$] (7,2)
        (7,2) to[R, *-*, l=$R_4$] (7,4)
        (7,4) to[R, *-, l=$R_5$] (10,4)
        (7,2) to[R, *-*, l=$R_6$] (10,2)
        (10,4) to[short] (10,0)
        (10,0) to[short] (6,0)
        (6,0) to [short, *-] (6,1)
        (6,0) to [short, *-] (6,-1)
        (6,1) to [R, l=$R_7$] (3,1)
        (6,-1) to [R, l=$R_8$] (3,-1)
        (3,1) to[short, -*] (3,0)
        (3,-1) to[short, -*] (3,0)
        (3,0) to[short] (0,0)
        ;
    \end{circuitikz}
\end{center}
\centerline {\huge{Riešenie metodou postupného zjedodušovania}}\\
Označíme si uzly pre transfiguráciu (trojuholník $\Rightarrow$ hviezda):\\
\begin{center}
\begin{circuitikz}
        \draw
        (0,4) to[dcvsource, v_=$U_1$] (0,2)
        (0,2) to[dcvsource, v_=$U_2$] (0,0)
        (0,4) to[short] (1,4)
        (1,4) to[R, *-, l=$R_{12}$] (4,4)
        (4,4) to[R, l=$R_A$] (7,4)
        (1,4) to[short, *-] (1,2)
        (1,2) to[R, l=$R_3$] (4,2)
        (4,2) to[R, l=$R_B$] (7,2)
        (7,3) to[R, *-, l=$R_C$] (10,3)
        (7,4) to[short] (7,2)
        (10,3) to[short] (10,0)
        (10,0) to[short] (6,0)
        (6,0) to [R, l=$R_{78}$] (3,0)
        (3,0) to[short] (0,0)
        ;
    \end{circuitikz}
\end{center}\\
Prevedieme transfiguráciu a spojenie $R_1$ a $R_2$:\\
$$R_{12} = R_1 + R_2 = 420 + 980 = 1400$$
$$R_A = \frac{R_4 * R_5}{R_4 + R_6 + R_5} = \frac{280*310}{280+710+310} = \frac{863}{13}\Omega$$
$$R_B = \frac{R_4 * R_6}{R_4 + R_6 + R_5} = \frac{280*710}{280+710+310} = \frac{1988}{13}\Omega$$
$$R_C = \frac{R_6 * R_5}{R_4 + R_6 + R_5} = \frac{710*310}{280+710+310} = \frac{2201}{13}\Omega$$
Sériove spojenie $R_{12}$ s $R_A$ a $R_3$ s $R_B$ a Paralelne spojenie $R_7$ s $R_8$
$$R_{12A} = R_{12} + R_A = 1400 + \frac{863}{13} = \frac{19063}{13}\Omega$$
$$R_{3B} = R_3 + R_A = 330 + \frac{1988}{13} = \frac{6278}{13}\Omega$$
$$R_{78} = \frac{R_7 * R_8}{R_7 + R_8} = \frac{240 * 200}{240 + 200} = \frac{1200}{11}\Omega$$
\begin{center}
 \begin{circuitikz}
        \draw
        (0,4) to[dcvsource, v_=$U_1$] (0,2)
        (0,2) to[dcvsource, v_=$U_2$] (0,0)
        (0,4) to[R, l=$R_{12A3B}$] (7,4)
        (7,4) to[R, l=$R_C$] (10,4)
        (10,4) to[short] (10,0)
        (10,0) to [R, l=$R_{78}$] (0,0)
        ;
    \end{circuitikz}
\end{center}
Paralelne spojenie $R_{12A}$ s $R_{3B}$ a spojenie zdrojov napätia
$$R_{123AB} = \frac{R_{12A} * R_{3B}}{R_{12A} + R_{3B}} = \frac{\frac{19063}{13} * \frac{6278}{13}}{\frac{19063}{13} + \frac{6278}{13}} = 363,2833\Omega$$
$$U_{12} = U_1 + U_2 = 105 + 85 = 190V$$
Sériové spojenie $R_{123AB}$ s $R_{78}$ s $R_{C}$
$$R_{EKV} = R_{123AB} + R_{78} + R_{C} = 363,2833 + \frac{1200}{11} + \frac{2201}{13} = 641,6819$$
\begin{center}
\begin{circuitikz}
        \draw
        (0,4) to[dcvsource, v_=$U$] (0,0)
        (0,4) to[R, l=$R_{EKV}$] (10,4)
        (10,4) to[short] (10,0)
        (10,0) to [short] (0,0)
        ;
    \end{circuitikz}
\end{center}
Celkový prúd $I$:
$$I = \frac{U}{R_{EKV}} = \frac{190}{641,6819} = 0,2961A$$
Teraz môžme spätne dopočítať prúd a napätie na rezistore $R_3$
$$U_{123AB} = I * R_{123AB} = 0,2961 * 363,2833 = 107,5682V$$
$$I_{R3} \equiv I_{RB} \equiv I_{R3B} = \frac{U_{123AB}}{R_{3B}} = \frac{107,5682}{\frac{6278}{13}} = 0,22274A$$
$$U_{R3} = I * R_3 = 0.2227 * 330 = 73,5055V$$
\newpage
\section*{2.}
Zadanie:\\
Stanovte $U_{R_1}$ a $I_{R_1}$. Použite metódu Théveninovej vety.
\begin{table}[ht]
    \centering
    \begin{tabular}{|[|c|c|c|c|c|c|]}
    \hline
    sk.&U [V]&R_1 [\Omega]&R_2 [\Omega]&R_3 [\Omega]&R_4 [\Omega]&R_5 [\Omega]\\
    \hline
    C&200&70&220&630&240&450\\
    \hline
    \end{tabular}
\end{table}
\newline
\begin{center}
    \begin{circuitikz}
        \draw
        (2,4) to[dcvsource, v=$U$] (2,1)
        (2,4) to[R, l=$R_5$] (0,4)
        (0,4) to[short] (0,5)
        (0,4) to[R, *-*, l=$R_2$] (-2,4)
        (0,5) to[R, l=$R_1$] (-2,5)
        (0,4) to[R, -*, l=$R_4$] (0,1)
        (0,1) to[R, l=$R_3$] (-2,1)
        (-2,1) to[short] (-2,5)
        (0,1) to[short] (2,1)
        ;
    \end{circuitikz}
\end{center}
\centerline{\huge{Použijeme metódu Théveninovej vety}}
Vypočítame si prvú smyčku :
$$-450I_1 - 240I_2 + 200 = 0$$
$$-45I_1 - 24I_2 + 20 = 0$$
Vypočítame si druhú smyčku :\\
$$-220(I_1-I_2)-630(I_1-I_2) + 240I_2 = 0$$
$$-220I_1 + 220I_2 - 630I_1 + 630I_2 + 240I_2 = 0$$
$$-850I_1 + 1090I_2 = 0$$
$$-85I_1 + 109I_2 = 0$$
Zistíme $I_2$:\\
$$45I_1 + 24I_2 = 20$$
$$-85I_1 + 109I_2 = 0$$ 
$$3825I_1 + 2040I_2 = 1700$$
$$-3825I_1 + 4905I_2 = 0 $$
$$6945I_2 = 1700$$
$$I_2 = 0,2448A$$
Výpočet $I_1$:\\
$$45I_1 + 24*0,2448 = 20$$
$$45I_1 + 5,8752 = 20$$
$$45I_1 = 14,1248$$
$$I_1 = 0,3139A$$
\begin{center}
    \begin{circuitikz}
        \draw
        (2,4) to[dcvsource, v=$U$] (2,1)
        (2,4) to[R, l=$R_5$] (0,4)
        (0,4) to[short, -*] (0,5)
        (0,4) to[R, *-*, l=$R_2$] (-2,4)
        (0,4) to[R, -*, l=$R_4$] (0,1)
        (0,1) to[R, l=$R_3$] (-2,1)
        (-2,1) to[short, -*] (-2,5)
        (0,1) to[short] (2,1)
        ;
        \draw[thin, ->, >=triangle 45] (1,2.8)node{$I_1$}  ++(-60:0.5) arc (-60:170:0.5);
        \draw[thin, ->, >=triangle 45] (-1,2.8)node{$I_1 - I_2$}  ++(-60:0.5) arc (-60:170:0.5);
    \end{circuitikz}
\end{center}
Odstránime cielený rezistor a počítame ďalej\\
Výpočet $V_{th}$:\\
$$-V_{th} + 220(I_1-I_2) = 0$$
$$V_{th} = 220(I_1-I_2)$$
$$V_{th} = 220 * 0,0691$$
$$V_{th} = 15,202V$$
Zjednodušíme obvod:\\
$$R_{345} = R_3 + \frac{R_4*R_5}{R_4+R_5}$$
$$R_{345} = \frac{18090}{23}$$
$$R_{th} = \frac{\frac{18090}{23}*220}{\frac{18090}{23}+220}$$
$$R_{th} = 171,9136\Omega$$
Vypočítame prúd I:\\
$$I_{R1} = \frac{V_{th}}{R_{th}+R_l}$$
$$I_{R1} = 0,0628A$$
A ako posledné vypočítame napätie U:\\
$$U_{R1} = R*I$$
$$U_{R1} = 4,396V$$
\newpage
\section*{3.}
Zadanie:\\
Stanovte napätie $U_{R_{3}}$ a prúd $I_{R_{3}}$.
Použite metódu uzlových napätí ($U_A$, $U_B$, $U_C$)\\
\begin{table}[ht]
    \centering
    \begin{tabular}{|c|c|c|c|c|c|c|c|c|}
        \hline
        sk. & $U$~[V] & $I_{1}$~[A] & $I_{2}$~[A] & $R_{1}~[\Omega]$ & $R_{2}~[\Omega]$ & $R_{3}~[\Omega]$ & $R_{4}~[\Omega]$ & $R_{5}~[\Omega]$ \\
        \hline
        B&150&0.7&0.8&49&45&61&34&34\\
        \hline
    \end{tabular}
\end{table}
\begin{center}
    \begin{circuitikz}
		\draw (0, -4) to[dcvsource=$U$] (0, -2) to[R=$R_{1}$] (0, 0) -- (2, 0) to[R=$R_{2}$, v=$U_{A}$, *-*] (2, -4) -- (0, -4);
		\draw (2, 0) -- (2, 2) to[ioosource=$I_{1}$] (6, 2) -- (6, 0) to[R=$R_{3}$, i=$I_{R_{3}}$, v=$U_{R_{3}}$, *-*] (2, 0);
		\draw (6, 0) to[R=$R_{4}$] (6, -4) -- (8, -4) to[ioosource=$I_{2}$] (8, 0) -- (6, 0);
		\draw (6, -4) to[R=$R_{5}$, v=$U_{C}$, *-*] (2, -4);
		\draw (6, 0) to[open, v=$U_{B}$] (2, -4);
	\end{circuitikz}
\end{center}
Označíme si jednotlivé uzly a zapíšeme rovnicu pre každý uzol
$$A:I_{R1}-I_1+I_{R3}-I_{R2}=0$$
$$B:I_1-I_{R3}+I_2-I_{R4}=0$$
$$C:I_{R4}-I_2-I_{R5}=0$$
Vyjadríme si prúdy na odporoch podľa ohmovho zákona
$$A:-I_1+\frac{U-U_A}{R_1}+\frac{U_B-U_A}{R3}-\frac{U_A}{R_2}=0$$
$$B:I_1-\frac{U_B-U_A}{R_3}+I_2-\frac{U_B-U_C}{R_4}=0$$
$$C:-I_2-\frac{U_C}{R_5}+\frac{U_B-U_C}{R4}=0$$
Dosadíme si čísla do rovníc
$$A:-0,7+\frac{150-U_A}{49}+\frac{U_B-U_A}{61}-\frac{U_A}{45}=0$$
$$B:0,7-\frac{U_B-U_A}{61}+0,8-\frac{U_B-U_C}{34}=0$$
$$C:-0,8-\frac{U_C}{34}+\frac{U_C-U_C}{34}=0$$
Po odstránení zlomkov a sčítaní rovnakých neznámych dostaneme výsledne rovnice v tvare
$$A:7939U_A-2205U_B=317596,5$$
$$B:34U_A-95U_B+61U_C=-3111$$
$$C:{U_B}-2U_C=27,2$$
Po výpočte 3 rovníc o 3 neznámych Cramerovým pravidlom dostaneme výsledné hodnoty
$$U_A=58,3750$$
$$U_B=66,1418$$
$$U_C=19,4709$$
Vieme, že $U_{R3}=U_B-U_A$ takže
$$U_{R3}=7,7688V$$
A ako posledné vypočítame $I_{R3}$
$$I_{R3}=\frac{U_{R3}}{R_3}$$
$$I_{R3}=0,1273A$$
\newpage
\section*{4.}
Pre napájacie napätie platí: $U_1=U_1*sin(2\pi ft), u_2=U_2*sin(2\Pi ft).$
Vo vzťahu pre napätie $u_{c2}=U_{c2}*sin(2\pi ft+\lambda_{c2})$ určte $|U_{c2}|$ a $\lambda_{c2}$. Použite metódu smyčkových prúdov.
\begin{table}[ht]
    \centering
    \begin{tabular}{|c|c|c|c|c|c|c|c|c|c|c|}
    \hline
    sk. &U_1[V] &U_2[V] &R_1[$\Omega$] &R_2[$\Omega$] &R_3[$\Omega$] &L_1[mH] &L_2[mH] &C_1[$\mu$F] &C_2[$\mu$F] &f[Hz]\\
    \hline
    D&45&50&13&15&13&180&90&210&75&85\\
    \hline
    \end{tabular}
\end{table} \\
\begin{figure}[!h]
	\centering
	\begin{circuitikz}
		\draw (0, 0) to[sV=$u_{1}$] (0, -2) -- (4, -2) to[R=$R_{3}$] (6, -2) -- (6, 0);
		\draw (0, 0) -- (0, 2) to[C=$C_{1}$] (4, 2) to[R=$R_{1}$] (6, 2) to[sV=$u_{2}$] (6, 0);
		\draw (0, 0) to[L=$L_{1}$, *-] (2, 0) to[R=$R_{2}$] (4, 0) to[L=$L_{2}$, -*] (6, 0);
		\draw (4, 0) to[C=$C_{2}$, v=$u_{C_{2}}$, i=$i_{C_{2}}$, *-*] (4, -2);
	\end{circuitikz}
\end{figure}\\
Vyjadríme si impedanciu cievok a kondenzátorov
$$Z_{c1}=-\frac{j}{\omega*C_1}$$
$$Z_{c1}=-8,9127j\Omega$$
$$Z_{c2}=-\frac{j}{\omega*C_2}$$
$$Z_{c2}=-24,9377j\Omega$$
$$Z_{L1}=j*\omega*L_1$$
$$Z_{L1}=96,1327j\Omega$$
$$Z_{L2}=j*\omega*L_2$$
$$Z_{L2}=48,0664j\Omega$$
$$U_1=U_1*sin(2\pi f\frac{\pi}{2\omega})$$
$$U_1=U_1*sin(90)$$
Zostavime Rovnice pre smyčky
$$A:-U_1+I_A*Z_{L1}+I_A*R_2+I_A+Z_{c2}-I_B*R_2-I_B*Z_{L1}-I_C*Z_{c2}=0$$
$$B:U2+I_B*Z_{L2}+I_B*R_2+I_B*Z_{L1}+I_B*Z_{c1}+I_B*R_1-I_A*Z_{L1}-I_A*R_2-I_C*Z_{l2}=0$$
$$C:I_C*R_3+I_C*Z_{c2}+I_C*Z_{L2}-I_A*Z_{c2}-I_B*Z_{L2}=0$$
Napätie na zdroji dáme na druhú stranu a vyjmeme $I_A,I_B,I_C$ pred zátvorku a usporiadame
$$I_A(Z_{L1}+R_2+Z_{c2})-I_B(R_2+Z_{L1})-I_C(Z_{c2})=U1$$
$$-I_A(Z_{L1}-R_2)+I_B(Z_{L2}+R_2+Z_{L1}+Z_{c1}+R_1)-I_C(Z_{L2})=-U2$$
$$-I_A(Z_{c2})-I_B(Z_{c2})+I_C(R_3+Z_{c2}+Z_{L2})=0$$
Zostavíme maticu a vypočítame $I_A,I_C,I_{c2}$
\begin{center}
    \begin{pmatrix}
      15 + 71,195j & -(15 + 96,1327j) & 24,9377j\\
     -(15 + 96,1327j) & 28 + 135,2864j & -48,0664j\\
     24,9377j & -48,0664j & 13 + 23,1287j\\
    \end{pmatrix}
    \begin{pmatrix}
      45\\
      -50\\
      0\\
    \end{pmatrix}
\end{center}

$$I_A=0,5563-0,9889j$$
$$I_C=-0,4135+0,3571j$$
$$I_{c2}=I_A-I_C=0,9698-1,346j$$
Vypočítame $|U_{c2}|$ a $\varphi_{c2}$
$$U_{c2} = I_{c2} * Z_{c2} = (0,9698 - 1,346j) * (-24,9377j) = -33.5661 - 24.6634j$$
$$|U_{c2}| = \sqrt{(33.5661)^2 + (24,4389)^2} = 41,5205V$$
$$\varphi _{c2} = -arctg(\frac{ImgU_{c2}}{ReU_{c2}}) * \frac{\pi}{180} = -arctg(\frac{33.5661}{24.4389}) * \frac{\pi}{180}= -0,01643 rad$$
\newpage
\section*{Výsledky}


\vspace{50px}

\begin{center}
    \begin{tabular}{| c | c | c |}
        \hline
        \textbf{Př.} & \textbf{Sk.} & \textbf{Výsledeky} \\
        \hline
        1 & D & $U_{R_3} = 73,5055V, I_{R_3} = 0,2227A$ \\
        \hline
        2 & C & $U_{R_1} = 4,396V, I_{R_1} = 0,0628A$ \\
        \hline
        3 & B & $U_{R_3} = 7,7688V, I_{R_3} = 0,1273A$ \\
        \hline
        4 & D & $|U_{C_2}| = 41,5205V, \varphi_{C_2} = -0,01643rad$ \\
        \hline
        5 & C & $??$ \\
        \hline
    \end{tabular}
\end{center}
\end{document}